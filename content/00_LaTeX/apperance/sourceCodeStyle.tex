% -- Eigene Farben für den Quellcode
\definecolor{JavaLila}{rgb}{0.4,0.1,0.4}
\definecolor{JavaGruen}{rgb}{0.3,0.5,0.4}
\definecolor{JavaBlau}{rgb}{0.0,0.0,1.0}
\definecolor{ABAPKeywordsBlue}{HTML}{6000ff}
\definecolor{ABAPCommentGrey}{HTML}{808080}
\definecolor{ABAPStringGreen}{HTML}{4da619}
\definecolor{PyKeywordsBlue}{HTML}{0000AC}
\definecolor{PyCommentGrey}{HTML}{808080}
\definecolor{PyStringGreen}{HTML}{008080}
% -- Farben für ABAP CDS
\definecolor{CDSString}{HTML}{FF8C00}
\definecolor{CDSKeywords}{HTML}{6000ff}
\definecolor{CDSAnnotation}{HTML}{00BFFF}
\definecolor{CDSComment}{HTML}{808080}
\definecolor{CDSFunc}{HTML}{FF0000}
% -- Farben für C++
\definecolor{C++Lila}{HTML}{A640FF}
\definecolor{C++Orange}{HTML}{D69D85}
\definecolor{C++Blau}{HTML}{569CD6}
\definecolor{C++DunkelBlau}{HTML}{0E4583}
\definecolor{C++Mint}{HTML}{B8D7A3}
\definecolor{C++Creme}{HTML}{DCDCAA}
\definecolor{C++Gruen}{HTML}{57A64A}
\definecolor{C++Tuerkis}{HTML}{4EC9B0}

% -- Default Listing-Styles

\lstset{
	% Das Paket "listings" kann kein UTF-8. Deswegen werden hier 
	% die häufigsten Zeichen definiert (ä,ö,ü,...)
	literate=%
		{á}{{\'a}}1 {é}{{\'e}}1 {í}{{\'i}}1 {ó}{{\'o}}1 {ú}{{\'u}}1
		{Á}{{\'A}}1 {É}{{\'E}}1 {Í}{{\'I}}1 {Ó}{{\'O}}1 {Ú}{{\'U}}1
		{à}{{\`a}}1 {è}{{\`e}}1 {ì}{{\`i}}1 {ò}{{\`o}}1 {ù}{{\`u}}1
		{À}{{\`A}}1 {È}{{\'E}}1 {Ì}{{\`I}}1 {Ò}{{\`O}}1 {Ù}{{\`U}}1
		{ä}{{\"a}}1 {ë}{{\"e}}1 {ï}{{\"i}}1 {ö}{{\"o}}1 {ü}{{\"u}}1
		{Ä}{{\"A}}1 {Ë}{{\"E}}1 {Ï}{{\"I}}1 {Ö}{{\"O}}1 {Ü}{{\"U}}1
		{â}{{\^a}}1 {ê}{{\^e}}1 {î}{{\^i}}1 {ô}{{\^o}}1 {û}{{\^u}}1
		{Â}{{\^A}}1 {Ê}{{\^E}}1 {Î}{{\^I}}1 {Ô}{{\^O}}1 {Û}{{\^U}}1
		{œ}{{\oe}}1 {Œ}{{\OE}}1 {æ}{{\ae}}1 {Æ}{{\AE}}1 {ß}{{\ss}}1
		{ű}{{\H{u}}}1 {Ű}{{\H{U}}}1 {ő}{{\H{o}}}1 {Ő}{{\H{O}}}1
		{ç}{{\c c}}1 {Ç}{{\c C}}1 {ø}{{\o}}1 {å}{{\r a}}1 {Å}{{\r A}}1
		{€}{{\euro}}1 {£}{{\pounds}}1 {«}{{\guillemotleft}}1
		{»}{{\guillemotright}}1 {ñ}{{\~n}}1 {Ñ}{{\~N}}1 {¿}{{?`}}1,
	breaklines=true,        % Breche lange Zeilen um 
	breakatwhitespace=true, % Wenn möglich, bei Leerzeichen umbrechen
	% Symbol für Zeilenumbruch einfügen
	prebreak=\raisebox{0ex}[0ex][0ex]{\ensuremath{\rhookswarrow}},
	postbreak=\raisebox{0ex}[0ex][0ex]{\ensuremath{\rcurvearrowse\space}},
	tabsize=4,                                 % Setze die Breite eines Tabs
	basicstyle=\ttfamily\small,                % Grundsätzlicher Schriftstyle
	columns=fixed,                             % Besseres Schriftbild
	numbers=left,                              % Nummerierung der Zeilen
	%frame=single,                             % Umrandung des Codes
	showstringspaces=false,                    % Keine Leerzeichen hervorheben
	keywordstyle=\color{blue},
	ndkeywordstyle=\bfseries\color{darkgray},
	identifierstyle=\color{black},
	commentstyle=\itshape\color{JavaGruen},   % Kommentare in eigener Farbe
	stringstyle=\color{JavaBlau},             % Strings in eigener Farbe,
	captionpos=b                             % Bild*unter*schrift
}

% ---- Eigener JAVA-Style für den Quellcode
\renewcommand{\ttdefault}{pcr}               % Schriftart, welche auch fett beinhaltet
\lstdefinestyle{EigenerJavaStyle}{
	language=Java,                             % Syntax Highlighting für Java
	%frame=single,                             % Umrandung des Codes
	keywordstyle=\bfseries\color{JavaLila},    % Keywords in eigener Farbe und fett
	commentstyle=\itshape\color{JavaGruen},    % Kommentare in eigener Farbe und italic
	stringstyle=\color{JavaBlau}               % Strings in eigener Farbe
}

% ---- Eigener ABAP-Style für den Quellcode
\renewcommand{\ttdefault}{pcr}
\lstdefinestyle{EigenerABAPStyle}{
	language=[R/3 6.10]ABAP,
	morestring=[b]\|,                          % Für Pipe-Strings
	morestring=[b]\`,                          % für Backtick-Strings
	keywordstyle=\bfseries\color{ABAPKeywordsBlue},
	commentstyle=\itshape\color{ABAPCommentGrey},
	stringstyle=\color{ABAPStringGreen},
	tabsize=2,
	morekeywords={
		types,
		@data,
		as,
		lower,
		start,
		selection,
		order,
		by,
		inner,
		join,
		key,
		end,
		cast
	}
}

% ---- Eigener Python-Style für den Quellcode
\renewcommand{\ttdefault}{pcr}
\lstdefinestyle{EigenerPythonStyle}{
	language=Python,
	columns=flexible,
	keywordstyle=\bfseries\color{PyKeywordsBlue},
	commentstyle=\itshape\color{PyCommentGrey},
	stringstyle=\color{PyStringGreen}
}

%----- ABAP-CDS-View language
\lstdefinelanguage{ABAPCDS}{
	sensitive=false,
	%Keywords
	morekeywords={define,
		view,
		as,
		select,
		from,
		inner,
		join,
		on,
		key,
		case,
		when,
		then,
		else,
		end,
		true,
		false,
		cast,
		where,
		and,
		distinct,
		group,
		by,
		having,
		min,
		sum,
		max,
		count,
		avg
	},
	%Methoden
	morekeywords=[2]{
		div,
		currency\_conversion,
		dats\_days\_between,
		concat\_with\_space,
		dats\_add_days,
		dats\_is\_valid,
		dats\_add\_months,
		unit\_conversion,
		division,
		mod,
		abs,
		floor,
		ceil,
		round,
		concat,
		replace,
		substring,
		left,
		right,
		length
	},
	morecomment=[s][\color{CDSAnnotation}]{@}{:},
	morecomment=[l][\itshape\color{CDSComment}]{//},
	morecomment=[s][\itshape\color{CDSComment}]{/*}{*/},
	morestring=[b][\color{CDSString}]',
	keywordstyle=\bfseries\color{CDSKeywords},
	keywordstyle=[2]\color{CDSFunc}
}


% ---- Eigener XSD-Style für den Quellcode
\renewcommand{\ttdefault}{pcr}
\lstdefinestyle{EigenerXSDStyle}{
	language=XML,                     
	stringstyle=\color{JavaGruen},
	morekeywords={
		encoding,
		xs:schema,
		xs:element,
		xs:complexType,
		xs:sequence,
		xs:attribute,
		xs:simpleContent,
		xs:extension base, 
		encoding,
		schema,
		element,
		complexType,
		sequence,
		attribute,
		simpleContent,
		extension base
	}
}

% ---- Eigener C++-Style für den Quellcode
\renewcommand{\ttdefault}{pcr}               % Schriftart, welche auch fett beinhaltet
\lstdefinestyle{EigenerC++Style}{
	language=C++,			% Standardsprache des Quellcodes
    columns=flexible,
	morekeywords={nullptr},
	% Enumerators (Mint)
	morekeywords=[2]{},
    % Methoden (Cremefarben)
	morekeywords=[3]{},
	% Extra Keywords (Lila)
	morekeywords=[4]{},
	% Noch mehr Keywords (Gruen)
	morekeywords=[5]{},
    keywordstyle=\bfseries\color{C++DunkelBlau},
    keywordstyle=[2]\color{C++Mint},
    keywordstyle=[3]\color{C++Creme},
    keywordstyle=[4]\color{C++Lila},
    keywordstyle=[5]\color{C++Tuerkis},
	commentstyle=\itshape\color{C++Gruen},
	stringstyle=\color{C++Orange}
}
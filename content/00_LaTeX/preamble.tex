\documentclass[
fontsize=12pt,          	 % Leitlinien sprechen von Schriftgröße 12.
paper=A4,
twoside=false,
listof=totoc,            		% Tabellen- und Abbildungsverzeichnis ins Inhaltsverzeichnis
index=totoc,				  %Stichwortverzeichnis wird in das Inhaltsverzeichniss aufgenommen
bibliography=totoc,		  % Literaturverzeichnis ins Inhaltsverzeichnis aufnehmen
titlepage,               % Titlepage-Umgebung anstatt \maketitle
headsepline,             % horizontale Linie unter Kolumnentitel
abstracton,              % Überschrift einschalten, Abstract muss in {abstract}-Umgebung stehen
ngerman
]{scrreprt}                  % Verwendung von KOMA-Report

\usepackage[onehalfspacing]{setspace}        % Zeilenabstand \singlespacing, \onehalfspaceing, \doublespacing
\usepackage{scrlayer-scrpage} 	   % Gestaltung von Fuß- und Kopfzeilen

\usepackage{titletoc}          			  % Anpassungen am Inhaltsverzeichnis
\contentsmargin{0.75cm}       		  % Abstand im Inhaltsverzeichnis zw. Punkt und Seitenzahl

% Paket für Quellen
\usepackage[
backend = biber,                % Verweis auf biber
language = auto,
style = numeric,                % Nummerierung der Quellen mit Zahlen
sorting = none,                 % none = Sortierung nach der Erscheinung im Dokument
sortcites = true,               % Sortiert die Quellen innerhalb eines cite-Befehls
block = space,                  % Extra Leerzeichen zwischen Blocks
hyperref = true,                % Links sind klickbar auch in der Quelle
%backref = true,                % Referenz, auf den Text an die zitierte Stelle
bibencoding = auto,
giveninits = true,              % Vornamen werden abgekürzt
doi=false,                      % DOI nicht anzeigen
isbn=false,                     % ISBN nicht anzeigen
alldates=short                  % Datum immer als DD.MM.YYYY anzeigen
]{biblatex}
\addbibresource{content/sources.bib}
\setcounter{biburlnumpenalty}{3000}     % Umbruchgrenze für Zahlen
\setcounter{biburlucpenalty}{6000}      % Umbruchgrenze für Großbuchstaben
\setcounter{biburllcpenalty}{9000}      % Umbruchgrenze für Kleinbuchstaben
\DeclareNameAlias{default}{family-given}  % Nachname vor dem Vornamen
\AtBeginBibliography{\renewcommand{\multinamedelim}{\addslash\space
	}\renewcommand{\finalnamedelim}{\multinamedelim}}  % Schrägstrich zwischen den Autorennamen
\DefineBibliographyStrings{german}{
	urlseen = {Einsichtnahme:},                      % Ändern des Titels von "besucht am"
}
\usepackage[babel,german=quotes]{csquotes}         % Deutsche Anführungszeichen + Zitate


% Unterstützung für Deutsch einbinden
\usepackage[ngerman]{babel}

% Pakete für die Darstellung von mathematischen Symbolen und Eingabemöglichkeit von Umlauten
\usepackage[T1]{fontenc}
\usepackage[utf8]{inputenc}
\usepackage[activate]{microtype}       % Trennung von Wörtern wird besser umgesetzt
\usepackage{lmodern}         % Nicht-gerasterte Schriftarten (bei MikTeX erforderlich)

% Pakete für Mathematik
\usepackage[sumlimits, intlimits]{amsmath}  % Bei Integralen und Summen werden die Zahlen oben drüber geschrieben 
\usepackage{amsmath}    % Erweiterung vom Mathe-Satz
\usepackage{amssymb}
\usepackage{amsfonts}
\usepackage{MnSymbol}   % Für Symbole, die in amssymb nicht enthalten sind.
\usepackage{array}
\usepackage{float}
\usepackage{mathtools}
\usepackage{mathrsfs}

% Pakete für Abkürzungen
\usepackage[printonlyused]{acronym}	%Abkürzungen werden nur gedruckt wenn sie gebraucht wurden

% Pakete für Tabellen
\usepackage{booktabs}  % Für schönere Tabellen. Enthält neue Befehle wie \midrule
\usepackage{multirow}  % Mehrzeilige Tabellen
\usepackage{siunitx}   % Für SI Einheiten und das Ausrichten Nachkommastellen
\sisetup{locale=DE, range-phrase={~bis~}, output-decimal-marker={,}} % Damit ein Komma und kein Punkt verwendet wird.
\usepackage{xfrac} % Für siunitx Option "fraction-function=\sfrac"
\setlength{\tabcolsep}{10pt}					% Abstand des Inhalts zur Linie
\renewcommand{\arraystretch}{1.25}		%Abstand der Zeilen wird erhöht 1.0 ist default

% Pakete für Bilder
\usepackage{wrapfig}
\usepackage{graphicx}
\usepackage{caption}

% Pakete für Definitionsboxen
\usepackage{amsthm}                     % Liefert die Grundlagen für Theoreme
\usepackage[framemethod=tikz]{mdframed} % Boxen für die Umrandung
\input{content/00_LateX/highlightBoxes}  % Weitere Details sind ausgelagert

% Pakete Für Todo Notes
\usepackage{todonotes}
\setlength {\marginparwidth }{2cm}      % Abstand für Todo Notizen

% Pakete für Aufzählungen
\usepackage{enumitem}

% Pakete für Referenzen
\usepackage{varioref}
\usepackage[                        % Klickbare Links (enth. auch "nameref", "url" Package)
hidelinks,                    			 % Blende die "URL Boxen" aus.
breaklinks=true              	    % Breche zu lange URLs am Zeilenende um
]{hyperref}

\hypersetup{%
	linktocpage=true, 		% Nicht der Text sondern die Seitenzahlen in Verzeichnissen klickbar
	bookmarksnumbered=true 	% Überschriftsnummerierung im PDF Inhalt anzeigen.
}
\usepackage[noabbrev]{cleveref}   %Kein erkennen von Abkürzungen 

% Quellcodevorlage
\usepackage{scrhack}                    % Hack zur Verw. von listings in KOMA-Script
\usepackage{listings}                   % Darstellung von Quellcode
\usepackage{xcolor}                     % Einfache Verwendung von Farben
\input{content/00_LateX/sourceCodeStyle}  % Weitere Details sind ausgelagert
\usepackage{algorithm}                  % Für Algorithmen-Umgebung (ähnlich wie lstlistings Umgebung)
\usepackage{algpseudocode}              % Für Pseudocode. Füge "[noend]" hinzu, wenn du kein "endif",
% etc. haben willst.
\makeatletter                           % Sorgt dafür, dass man @ in Namen verwenden kann.
% Ansonsten gibt es in der nächsten Zeile einen Compilefehler.
\renewcommand{\ALG@name}{Algorithmus}   % Umbenennen von "Algorithm" im Header der Listings.
\makeatother                            % Zeichen wieder zurücksetzen
\renewcommand{\lstlistingname}{Listing} % Erlaubt das Umbenennen von "Listing" in anderen Titel.

% Anhänge werden sortiert
\usepackage[titletoc]{appendix} %Fügt "Anhang" in TOC ein. 

% Paket für ein Stichwortverzeichnis
\usepackage{makeidx}
\makeindex

% Nur ein latex-Durchlauf für die Aktualisierung von Verzeichnissen nötig
\usepackage{bookmark} 

% Zeilenumbruch, Seitenränder und mehr
\usepackage[
%showframe,                % Ränder anzeigen lassen
left=2.7cm, right=2.5cm,
top=2.5cm,  bottom=2.5cm,
includeheadfoot
]{geometry}                 

% Schriftart wird geändert
\usepackage{lmodern}
\renewcommand*\familydefault{\sfdefault} 				%Serifenlose Schrift 

%Unterüberschriften 
\usepackage[list=true]{subcaption} %Unterbilder werden im Abbildungsverzeichniss dargestellt 


% Bildunterschriften werden etwas kleiner dargestellt
\addtokomafont{caption}{\small}

\addto\extrasngerman{%
	\def\equationautorefname~#1\null{Gleichung~(#1)\null}
	\def\lstnumberautorefname{Zeile}
	\def\lstlistingautorefname{Listing}
	\def\algorithmautorefname{Algorithmus}
	% Damit einheitlich "Abschnitt 1.2[.3]" verwendet wird und nicht "Unterabschnitt 1.2.3"
	% \def\subsectionautorefname{Abschnitt}
}

% ---- Abstand verkleinern von der Überschrift 
\renewcommand*{\chapterheadstartvskip}{\vspace*{.5\baselineskip}}

% Einzelne Wörter
\widowpenalty10000
\clubpenalty10000

% Paket für Blindtext
\usepackage{lipsum}

% Layout für die Absätze
\setlength{\parindent} {0.0em}
\setlength{\parskip} {1.5ex plus0.5ex minus0.5ex}   % Setzt den Abstand zwischen 2 Absätzen 

% Current file is content/00_LaTeX/preamble.tex

%------------------------------------ Informationen ------------------------------------%
% Bei diesem Dokument handelt es sich um die Präambel des Dokuments. 
% Strukur der Präambel:
% 01: Allgemeines
% 02: Mathematik
% 03: Verzeichnisse
% 04: Quellenverwaltung
% 05: Bilder
% 06: Tabellen
% 07: Listen
% 08: Listings
% 09: Eigene Umgebungen
% 10: Abkürzungen 
% 11: Referenzen
% 12: Variablen
% 13: Sonstige 


%------------------------------------ Allgemeines ------------------------------------%
% 1.	Dokumentenklasse
% 2.	Sprache und Umlaute
% 3.	Zeilenabstand
% 4.	Zeilenumbrüche und Seitenränder
% 5.	Absätze
% 6.	Schriftart
% 7.	Überschriften
% 8.	Automatische Seitenumbrüche von LaTeX

% -- Deklarierung der Dokumentenklasse scrreprt
\documentclass[
	fontsize=12pt,			% Leitlinien von der Schriftgröße 12.
	paper=A4,				% Papiergröße wird auf A4 festgelegt
	twoside=false,			% Zweiseitiger Druck wird deaktiviert
	listof=totoc,           % Tabellen- und Abbildungsverzeichnis in das Inhaltsverzeichnis
	index=totoc,			% Stichwortverzeichnis wird in das Inhaltsverzeichnis aufgenommen
	bibliography=totoc,		% Literaturverzeichnis wird in das Inhaltsverzeichnis aufnehmen
	titlepage,              % Titlepage-Umgebung anstatt \maketitle
	headsepline,            % Horizontale Linie unter Kolumnentitel
	hidelinks,				% Boxen um klickbare Links werden entfernt
	ngerman,				% Sprache wird auf Deutsch festgelegt
	english					% Englisch als weiter Sprache hinzufügen
]{scrreprt}             	% Verwendung von KOMA-Report


% -- Einstellung der Sprache des Dokuments und die Darstellung von mathematischen Symbolen und Eingabemöglichkeit von Umlauten
\usepackage[
	main = ngerman,							% Sprache wird auf Deutsch festgelegt
	english									% Englisch wird unterstützt
]{babel}									
\usepackage[T1]{fontenc}					% Voraussetzung für LaTeX Fonts in westeuropäischer Codierung
\usepackage[utf8]{inputenc}					% UTF8 wird unterstützt
\usepackage[activate]{microtype}      		% Trennung von Wörtern wird besser umgesetzt
\usepackage{lmodern}         				% Nicht-gerasterte Schriftarten (bei MikTeX erforderlich)
\usepackage[babel]{csquotes}  				% Anführungszeichen + Zitate


% -- Einstellungen am Zeilenabstand
\usepackage[onehalfspacing]{setspace}		% Zeilenabstand \singlespacing, \onehalfspaceing, \doublespacing
\usepackage{scrlayer-scrpage} 	   			% Paket für die Gestaltung von Fuß- und Kopfzeilen


% -- Einstellungen für den Zeilenumbruch, Seitenränder und mehr
\usepackage[
	left=2.7cm, 							% Ränder links werden eingestellt
	right=2.5cm,							% Ränder rechts werden eingestellt
	top=2.5cm,  							% Ränder oben werden eingestellt
	bottom=2.5cm,							% Ränder unten werden eingestellt
	includeheadfoot							% Schließt den Kopf und den Fuß in den druckbaren Bereich mit ein
]{geometry}                 


% Layout für die Absätze
\setlength{\parindent} {0.0em}
\setlength{\parskip} {1.5ex plus0.5ex minus0.5ex}   	% Setzt den Abstand zwischen 2 Absätzen 


% -- Einstellung der Schriftart
\usepackage[scaled=.95]{helvet}
\renewcommand*\familydefault{\sfdefault} 				% Serifenlose Schrift 


% ---- Abstand verkleinern von der Überschrift 
\renewcommand*{\chapterheadstartvskip}{\vspace*{.5\baselineskip}}


% -- Automatische Seitenumbrüche von LaTeX verbessern
\widowpenalty10000
\clubpenalty10000


%------------------------------------ Mathematik ------------------------------------%
% 1.	Erweiterungen des Mathe-Satzes
% 2.	SI-Einheiten

% -- Einstellungen für Mathematik
\usepackage[sumlimits, intlimits]{amsmath}  % Bei Integralen und Summen werden die Zahlen oben drüber geschrieben 
\usepackage{amssymb}						% Erweiterung vom Mathe-Satz durch Symbole	
\usepackage{amsfonts}						% Zusätzliche Schriftarten für den Mathe-Satz, sowie neue Symbole
\usepackage{MnSymbol}   					% Für Symbole, die in amssymb nicht enthalten sind.						
\usepackage{mathtools}						% Verbesserung der Anzeige von mathematischen Symbolen
\usepackage{xfrac}							% Erweiterung der Brüche im Mathemodus
\usepackage{mathrsfs}						% Raph Smith’s Formal Script Schriftart i Mathe-Modus


% -- Einstellungen der SI Einheiten werden festgelegt.
\usepackage{siunitx}
\sisetup{
	locale=DE, 								% Sprache wird auf Deutsch festgelegt
	range-phrase={~bis~}, 
	output-decimal-marker={,}				% Damit ein Komma und kein Punkt verwendet wird.
} 


%------------------------------------ Verzeichnisse ------------------------------------%
% 1.	Inhaltsverzeichnis
% 2.	Stichwortverzeichnis
% 3.	Verbesserung der Übersetzung
% 4.	Anhang
% 5.	Glossar

% -- Einstellungen am Inaltsverzeichnis
\usepackage{titletoc}				  		% Paket für die Anpassungen am Inhaltsverzeichnis und dem Anhang
\contentsmargin{0.75cm}       		  		% Abstand im Inhaltsverzeichnis zw. Punkt und Seitenzahl


% -- Einstellungen für das Stichwortverzeichnis
\usepackage{makeidx}						% Paket für das Stichwortverzeichnis
\makeindex							


% -- Verbesserung der Übersetzung
\usepackage{bookmark}						% Nur ein LaTeX-Durchlauf für die Aktualisierung von Verzeichnissen nötig


% -- Einstellungen für den Anhang
\usepackage[titletoc]{appendix} %Fügt "Anhang" in TOC ein. 


% -- Einstellungen für das Glossar
\usepackage[
	nonumberlist,							% Keine Seitenzahlen im Glossar
	toc										% Aufnahme in das Inhaltsverzeichniss 
]{glossaries}

%------------------------------------ Quellenverwaltung ------------------------------------%
% 1.	Einstellungen von Biblatex
% 2.	Speicherort der Bibtex-Datei
% 3.	Umbruchgrenzen im Literaturverzeichnis

% -- Einstellungen für Biblatex
\usepackage[
	backend = biber,                		% Verweis auf biber
	language = auto,						% Sprache wird automatisch festgelegt
	style = numeric-comp,                	% Nummerierung der Quellen mit Zahlen
	bibencoding = utf8,						% UTF8 wird in biblatex aktiviert
	sorting = none,                 		% none = Sortierung nach der Erscheinung im Dokument
	sortcites = true,               		% Sortiert die Quellen innerhalb eines cite-Befehls
	block = space,                  		% Extra Leerzeichen zwischen Blocks
	hyperref = true,                		% Links sind klickbar auch in der Quelle
	giveninits = true,              		% Vornamen werden abgekürzt
	doi=true,                      			% DOI anzeigen
	isbn=true,                     			% ISBN anzeigen
	alldates=short                  		% Datum immer als DD.MM.YYYY anzeigen
]{biblatex}

% -- Einstellung des Dateipfades der Bibtex-Datei 
% -- Für das example Projekt wird in der Datei runVariables.tex eine Weitere Datei eingebunden.
\addbibresource{content/sources.bib}		% Speicherort der Quellendatei


% -- Einstellung der Umbruchgrenzen im Literaturverzeichnis
\setcounter{biburlnumpenalty}{3000}     	% Umbruchgrenze für Zahlen
\setcounter{biburlucpenalty}{6000}      	% Umbruchgrenze für Großbuchstaben
\setcounter{biburllcpenalty}{9000}     		% Umbruchgrenze für Kleinbuchstaben
\DeclareNameAlias{default}{family-given}	% Nachname vor dem Vornamen


% -- Änderungen am Layout der Quellen im Literaturverzeichnis 
\AtBeginBibliography{										% Schrägstrich zwischen den Autorennamen
	\renewcommand{\multinamedelim}{\addslash\space} 		
	\renewcommand{\finalnamedelim}{\multinamedelim}
}  

\DefineBibliographyStrings{german}{							% Ändern des Titels von "besucht am":
	urlseen = {Einsichtnahme:},                      
}


%------------------------------------ Bilder ------------------------------------%
% 1.	Grundlegende Pakete für Bilder
% 2.	Bildunterschriften
% 3. 	Allgemeine Verbesserungen

% -- Benötigte Pakete für Bilder
\usepackage{wrapfig}						% Wrapfigureumgebung wird ermöglicht
\usepackage{graphicx}						% Standardpaket für Bilder
\usepackage{caption}						% Bildunterschriften werde eingebunden


% -- Einstellungen für Bilder
\graphicspath{{images/}} 					% Standardpfad für Bilder wird angepasst, sodass nur noch der Dateiname des Bildes angegeben werden muss


% -- Einstellungen für Unterüberschriften 
\addtokomafont{caption}{\small} 			% Bildunterschriften werden etwas kleiner dargestellt
\usepackage[list=true]{subcaption} 			% Subbilder des Paketes subifigure werden im Abbildungsverzeichniss dargestellt 


\usepackage{float} 							% Verbesserung der floating-Option bei Bildern durch hinzufügen von "H" als Option


%------------------------------------ Tabellen ------------------------------------%
% 1.	Grundlegende Pakete für Tabellen
% 2.	Zeilen- und Linienabstände 

% -- Benötigte Pakete für Tabellen
\usepackage{booktabs}  						% Für schönere Tabellen. Enthält neue Befehle wie \midrule
\usepackage{multirow}  						% Mehrzeilige Tabellen
\usepackage{longtable}						% Verbesserte Tabellenumgebung
\usepackage{array}							% Erweitert die Tabellenumgebung
\usepackage{makecell}						% Ermöglichen von Umbrüchen in Tabellen

% -- Einstellungen für Tabellen
\setlength{\tabcolsep}{10pt}				% Abstand des Inhalts zur Linie
\renewcommand{\arraystretch}{1.25}			% Abstand der Zeilen wird erhöht 1.0 ist default


%------------------------------------ Aufzählungen ------------------------------------%
% 1.	Grundlegende Pakete für Aufzählungen

% -- Benötigte Pakete für Aufzählungen
\usepackage{enumitem}		% Ermöglicht Listen, wie itemize und enumerate


%------------------------------------ Listings ------------------------------------%
% 1.	Grundlegende Pakete für Listings
% 2.	Importierung des SourceCodeStyles
% 3.	Sonstige Einstellungen

% -- Grundlegende Pakete für Listings
\usepackage{scrhack}                    					% Hack zur Verw. von listings in KOMA-Script
\usepackage{listings}                 					  	% Darstellung von Quellcode
\usepackage{xcolor}                     					% Einfache Verwendung von Farben
\usepackage{algorithm}                  					% Für Algorithmen-Umgebung (ähnlich wie lstlistings Umgebung)
\usepackage{algpseudocode}              					% Für Pseudocode. Füge "[noend]" hinzu, wenn du kein "endif", etc. haben willst.


% -- Importierung des SourceCodeStyles
% -- Default Farben für den Quellcode
\definecolor{DefaultGruen}{rgb}{0.3,0.5,0.4}
\definecolor{DefaultBlau}{rgb}{0.0,0.0,1.0}

% -- Default Listing-Style
\lstset{
	% Das Paket "listings" kann kein UTF-8. Deswegen werden hier 
	% die häufigsten Zeichen definiert (ä,ö,ü,...)
	literate=%
		{á}{{\'a}}1 {é}{{\'e}}1 {í}{{\'i}}1 {ó}{{\'o}}1 {ú}{{\'u}}1
		{Á}{{\'A}}1 {É}{{\'E}}1 {Í}{{\'I}}1 {Ó}{{\'O}}1 {Ú}{{\'U}}1
		{à}{{\`a}}1 {è}{{\`e}}1 {ì}{{\`i}}1 {ò}{{\`o}}1 {ù}{{\`u}}1
		{À}{{\`A}}1 {È}{{\'E}}1 {Ì}{{\`I}}1 {Ò}{{\`O}}1 {Ù}{{\`U}}1
		{ä}{{\"a}}1 {ë}{{\"e}}1 {ï}{{\"i}}1 {ö}{{\"o}}1 {ü}{{\"u}}1
		{Ä}{{\"A}}1 {Ë}{{\"E}}1 {Ï}{{\"I}}1 {Ö}{{\"O}}1 {Ü}{{\"U}}1
		{â}{{\^a}}1 {ê}{{\^e}}1 {î}{{\^i}}1 {ô}{{\^o}}1 {û}{{\^u}}1
		{Â}{{\^A}}1 {Ê}{{\^E}}1 {Î}{{\^I}}1 {Ô}{{\^O}}1 {Û}{{\^U}}1
		{œ}{{\oe}}1 {Œ}{{\OE}}1 {æ}{{\ae}}1 {Æ}{{\AE}}1 {ß}{{\ss}}1
		{ű}{{\H{u}}}1 {Ű}{{\H{U}}}1 {ő}{{\H{o}}}1 {Ő}{{\H{O}}}1
		{ç}{{\c c}}1 {Ç}{{\c C}}1 {ø}{{\o}}1 {å}{{\r a}}1 {Å}{{\r A}}1
		{€}{{\euro}}1 {£}{{\pounds}}1 {«}{{\guillemotleft}}1
		{»}{{\guillemotright}}1 {ñ}{{\~n}}1 {Ñ}{{\~N}}1 {¿}{{?`}}1,
	breaklines=true,        % Breche lange Zeilen um 
	breakatwhitespace=true, % Wenn möglich, bei Leerzeichen umbrechen
	% Symbol für Zeilenumbruch einfügen
	prebreak=\raisebox{0ex}[0ex][0ex]{\ensuremath{\rhookswarrow}},
	postbreak=\raisebox{0ex}[0ex][0ex]{\ensuremath{\rcurvearrowse\space}},
	tabsize=4,                                 % Setze die Breite eines Tabs
	basicstyle=\ttfamily\small,                % Grundsätzlicher Schriftstyle
	columns=fixed,                             % Besseres Schriftbild
	numbers=left,                              % Nummerierung der Zeilen
	%frame=single,                             % Umrandung des Codes
	showstringspaces=false,                    % Keine Leerzeichen hervorheben
	keywordstyle=\color{blue},
	ndkeywordstyle=\bfseries\color{darkgray},
	identifierstyle=\color{black},
	commentstyle=\itshape\color{DefaultGruen},   % Kommentare in eigener Farbe
	stringstyle=\color{DefaultBlau},             % Strings in eigener Farbe,
	captionpos=b                             % Bild*unter*schrift
}

% -- Laden der zusätzlichen Styles
% -- Eigene Farben für den Style
\definecolor{CDSString}{HTML}{FF8C00}
\definecolor{CDSKeywords}{HTML}{6000ff}
\definecolor{CDSAnnotation}{HTML}{00BFFF}
\definecolor{CDSComment}{HTML}{808080}
\definecolor{CDSFunc}{HTML}{FF0000}

% -- Eigener ABAP-CDS-View-Style für den Quellcode
\lstdefinelanguage{ABAPCDS}{
	sensitive=false,
	%Keywords
	morekeywords={define,
		view,
		as,
		select,
		from,
		inner,
		join,
		on,
		key,
		case,
		when,
		then,
		else,
		end,
		true,
		false,
		cast,
		where,
		and,
		distinct,
		group,
		by,
		having,
		min,
		sum,
		max,
		count,
		avg
	},
	%Methoden
	morekeywords=[2]{
		div,
		currency\_conversion,
		dats\_days\_between,
		concat\_with\_space,
		dats\_add_days,
		dats\_is\_valid,
		dats\_add\_months,
		unit\_conversion,
		division,
		mod,
		abs,
		floor,
		ceil,
		round,
		concat,
		replace,
		substring,
		left,
		right,
		length
	},
	morecomment=[s][\color{CDSAnnotation}]{@}{:},
	morecomment=[l][\itshape\color{CDSComment}]{//},
	morecomment=[s][\itshape\color{CDSComment}]{/*}{*/},
	morestring=[b][\color{CDSString}]',
	keywordstyle=\bfseries\color{CDSKeywords},
	keywordstyle=[2]\color{CDSFunc}
}
% -- Eigene Farben für den Style
\definecolor{ABAPKeywordsBlue}{HTML}{6000ff}
\definecolor{ABAPCommentGrey}{HTML}{808080}
\definecolor{ABAPStringGreen}{HTML}{4da619}

% -- Eigener ABAP-Style für den Quellcode
\renewcommand{\ttdefault}{pcr}
\lstdefinestyle{EigenerABAPStyle}{
	language=[R/3 6.10]ABAP,
	morestring=[b]\|,                          % Für Pipe-Strings
	morestring=[b]\`,                          % für Backtick-Strings
	keywordstyle=\bfseries\color{ABAPKeywordsBlue},
	commentstyle=\itshape\color{ABAPCommentGrey},
	stringstyle=\color{ABAPStringGreen},
	tabsize=2,
	morekeywords={
		types,
		@data,
		as,
		lower,
		start,
		selection,
		order,
		by,
		inner,
		join,
		key,
		end,
		cast
	}
}
% Current file is content/00_LaTeX/sourceCodeStyle/languages/C&C++.tex

% -- Eigene Farben für den Style
\definecolor{C++Lila}{HTML}{A640FF}
\definecolor{C++Orange}{HTML}{D69D85}
\definecolor{C++Blau}{HTML}{569CD6}
\definecolor{C++DunkelBlau}{HTML}{0E4583}
\definecolor{C++Mint}{HTML}{B8D7A3}
\definecolor{C++Creme}{HTML}{DCDCAA}
\definecolor{C++Gruen}{HTML}{57A64A}
\definecolor{C++Tuerkis}{HTML}{4EC9B0}

% -- Eigener C/++-Style für den Quellcode
\renewcommand{\ttdefault}{pcr}               % Schriftart, welche auch fett beinhaltet
\lstdefinestyle{EigenerC++Style}{
	language=C++,			% Standardsprache des Quellcodes
    columns=flexible,
	morekeywords={nullptr},
	% Enumerators (Mint)
	morekeywords=[2]{},
    % Methoden (Cremefarben)
	morekeywords=[3]{},
	% Extra Keywords (Lila)
	morekeywords=[4]{},
	% Noch mehr Keywords (Gruen)
	morekeywords=[5]{},
    keywordstyle=\bfseries\color{C++DunkelBlau},
    keywordstyle=[2]\color{C++Mint},
    keywordstyle=[3]\color{C++Creme},
    keywordstyle=[4]\color{C++Lila},
    keywordstyle=[5]\color{C++Tuerkis},
	commentstyle=\itshape\color{C++Gruen},
	stringstyle=\color{C++Orange}
}
% Current file is content/00_LaTeX/sourceCodeStyle/languages/Java.tex

% -- Eigene Farben für den Style
\definecolor{JavaLila}{rgb}{0.4,0.1,0.4}
\definecolor{JavaGruen}{rgb}{0.3,0.5,0.4}
\definecolor{JavaBlau}{rgb}{0.0,0.0,1.0}

% -- Eigener JAVA-Style für den Quellcode
\renewcommand{\ttdefault}{pcr}               % Schriftart, welche auch fett beinhaltet
\lstdefinestyle{EigenerJavaStyle}{
	language=Java,                             % Syntax Highlighting für Java
	%frame=single,                             % Umrandung des Codes
	keywordstyle=\bfseries\color{JavaLila},    % Keywords in eigener Farbe und fett
	commentstyle=\itshape\color{JavaGruen},    % Kommentare in eigener Farbe und italic
	stringstyle=\color{JavaBlau}               % Strings in eigener Farbe
}
% -- Eigene Farben für den Style


% -- Eigener JSON-Style für den Quellcode

% Current file is content/00_LaTeX/sourceCodeStyle/languages/Python.tex

%-- Eigene Farben für den Style
\definecolor{PyKeywordsBlue}{HTML}{0000AC}
\definecolor{PyCommentGrey}{HTML}{808080}
\definecolor{PyStringGreen}{HTML}{008080}

% -- Eigener Python-Style für den Quellcode
\renewcommand{\ttdefault}{pcr}
\lstdefinestyle{EigenerPythonStyle}{
	language=Python,
	columns=flexible,
	keywordstyle=\bfseries\color{PyKeywordsBlue},
	commentstyle=\itshape\color{PyCommentGrey},
	stringstyle=\color{PyStringGreen}
}
% -- Eigene Farben für den Style
\definecolor{XMLGruen}{rgb}{0.3,0.5,0.4}

% -- Eigener XML/XSD-Style für den Quellcode
\renewcommand{\ttdefault}{pcr}
\lstdefinestyle{EigenerXSDStyle}{
	language=XML,                     
	stringstyle=\color{XMLGruen},
	morekeywords={
		encoding,
		xs:schema,
		xs:element,
		xs:complexType,
		xs:sequence,
		xs:attribute,
		xs:simpleContent,
		xs:extension base, 
		encoding,
		schema,
		element,
		complexType,
		sequence,
		attribute,
		simpleContent,
		extension base
	}
}  	% Weitere Details sind ausgelagert


% -- Sonstige Einstellungen für die Quellcodevorlage
\makeatletter                          						% Sorgt dafür, dass man @ in Namen verwenden kann.
															% Ansonsten gibt es in der nächsten Zeile einen Compilefehler.
\renewcommand{\ALG@name}{Algorithmus}   					% Umbenennen von "Algorithm" im Header der Listings.
\makeatother                            					% Zeichen wieder zurücksetzen
\renewcommand{\lstlistingname}{Listing} 					% Erlaubt das Umbenennen von "Listing" in anderen Titel.

\addto\extrasngerman{%
	\def\equationautorefname~#1\null{Gleichung~(#1)\null}
	\def\lstnumberautorefname{Zeile}
	\def\lstlistingautorefname{Listing}
	\def\algorithmautorefname{Algorithmus}
	% Damit einheitlich "Abschnitt 1.2[.3]" verwendet wird und nicht "Unterabschnitt 1.2.3"
	% \def\subsectionautorefname{Abschnitt}
}


%------------------------------------ Eigene Umgebungen ------------------------------------%
% 1.	Definitionsboxen

% -- Einstellungen für Definitionsboxen
\usepackage{amsthm}                     				% Liefert die Grundlagen für Theoreme
\usepackage[framemethod=tikz]{mdframed} 				% Boxen für die Umrandung
% Current file is content/00_LaTeX/miscellanious/highlightBoxes.tex

% -- Grundsätzliche Definition zum Style
\newtheoremstyle{defi}
  {\topsep}         % Abstand oben
  {\topsep}         % Abstand unten
  {\normalfont}     % Schrift des Bodys
  {0pt}             % Einschub der ersten Zeile
  {\bfseries}       % Darstellung von der Schrift in der Überschrift
  {:}               % Trennzeichen zwischen Überschrift und Body
  {.5em}            % Abstand nach dem Trennzeichen zum Body Text
  {\thmname{#3}}    % Name in eckigen Klammern
\theoremstyle{defi}

% -- Definition zum Strich vor einem Text
\newmdtheoremenv[
  hidealllines = true,       % Rahmen komplett ausblenden
  leftline = true,           % Linie links einschalten
  innertopmargin = 0pt,      % Abstand oben
  innerbottommargin = 4pt,   % Abstand unten
  innerrightmargin = 0pt,    % Abstand rechts
  linewidth = 3pt,           % Linienbreite
  linecolor = gray!40,       % Linienfarbe
]{defStrich}{Definition}     % Name des Formats "defStrich"

% -- Definition zum Eck-Kasten um einen Text
\newmdtheoremenv[
  hidealllines = true,                                                  % Rahmen komplett ausblenden
  innertopmargin = 6pt,                                                 % Abstand oben
  linecolor = gray!40,                                                  % Linienfarbe
  singleextra={                                                         % Eck-Markierungen für die Definition
    \draw[line width=3pt,gray!50,line cap=rect] (O|-P) -- +(1cm,0pt);
    \draw[line width=3pt,gray!50,line cap=rect] (O|-P) -- +(0pt,-1cm);
    \draw[line width=3pt,gray!50,line cap=rect] (O-|P) -- +(-1cm,0pt);
    \draw[line width=3pt,gray!50,line cap=rect] (O-|P) -- +(0pt,1cm);
  }
]{defEckKasten}{Definition}                                             % Name des Formats "defEckKasten"  	% Weitere Details sind ausgelagert


%------------------------------------ Abkürzungen ------------------------------------%
% 1.	Abkürzungen

% -- Einstellungen für Abkürzungen
\usepackage{acro}	%Abkürzungen werden nur gedruckt wenn sie gebraucht wurden
\acsetup{ 
	list/display = used, 				% zeigt nur genutzte Abkürzungen 
	list/sort = true, 					% sortiert alphabetisch 
}
%Currently no working solution due to package update
\DeclareAcronym{f}{
  short = \ensuremath{f} ,
  long  = Frequenz
}
\DeclareAcronym{A}{
  short = \ensuremath{A} ,
  long  = Fläche
}
\DeclareAcronym{C}{
  short = \ensuremath{C} ,
  long  = Kapazität
}
\DeclareAcronym{F}{
  short = \ensuremath{F} ,
  long  = Kraft
}

% \begin{acronym}[WYSISWG] % längstes Kürzel wird verw. für den Abstand zw. Kürzel u. Text
	
	% Alphabetisch selbst sortieren - nicht verwendete Kürzel rausnehmen!
	\DeclareAcronym{AIR}{
  		short = AIR,
  		long  = Adobe Integrated Runtime
	}
	% \acro{AJAX}{Asynchronous Javascript and XML}
	\DeclareAcronym{AJAX}{
  		short = AJAX,
  		long  = Asynchronous Javascript and XML
	}
	% \acro{ANSI}{American National Standards Institute}
	% \acro{API}{Application Programming Interface}
	\DeclareAcronym{API}{
		short = API,
		long  = Application Programming Interface
	}
	% \acro{AR}{Augmented Reality}
	% \acro{BAPI}{Business Application Programming Interface}
	% \acro{BIOS}{Basic Input Output System}
	% \acro{CDMA}{Code Division Multiple Access}
	% \acro{HTTPS}{Hypertext Transfer Protocol Secure}
	\DeclareAcronym{HTTPS}{
		short = HTTPS,
		long  = Hypertext Transfer Protocol Secure
	}
	% \acro{ISBN}{Internationale Standardbuchnummer}
	\DeclareAcronym{ISBN}{
		short = ISBN,
		long  = Internationale Standardbuchnummer
	}
	% \acrodefplural{ISBN}[ISBNs]{Internationale Standardbuchnummern}
	% \acro{OLAP}{Online Analytical Processing}
	% \acro{ORDBMS}{Object-Relational DataBase Management System}
	% \acro{SDK}{Software Development Kit}
	% \acro{SEO}{Search Engine Optimization}
	% \acro{SSH}{Secure Shell}
	% \acro{UEFI}{Unified Extensible Firmware Interface}
	% \acro{USB}{Universal Serial Bus}
	% \acro{VLAN}{Virtual Local Area Network}
	% \acro{WYSISWG}{What You See Is What You Get}
	% \acro{XSL}{Extensible Stylesheet Language}
	
% \end{acronym}


%------------------------------------ Referenzen ------------------------------------%
% 1.	Varioref
% 2.	Hyperref
% 3.	Cleverev

% -- Varioref
\usepackage{varioref}


% -- Hyperref
\usepackage{hyperref}
\hypersetup{%
	linktocpage=true, 				% Nicht der Text sondern die Seitenzahlen in Verzeichnissen klickbar
	bookmarksnumbered=true 			% Überschriftsnummerierung im PDF Inhalt anzeigen.
}


% -- Cleverev
\usepackage[noabbrev]{cleveref}   	% Kein Erkennen von Abkürzungen 


%------------------------------------ Variablen ------------------------------------%
% 1.	Aufsetzten von Variablen

% -- Aufsetzen von Variablen
\usepackage{ifthen}
\usepackage{ifthen}
\newboolean{durchgehendeFussnoten}
\newboolean{sperrvermerkLangeVersion}
\newboolean{abgabeVersion}


%------------------------------------ Sonstiges ------------------------------------%
% 1.	Sonstige Pakete
% 2.	Sonstige Einstellungen

% Neue Überschrift für das Literaturverzeichnis auf Deutsch
\renewcaptionname{ngerman}{\bibname}{Literaturverzeichnis}
\newcaptionname{ngerman}{\lstlistlistingname}{Quellcodeverzeichnis}
\newcaptionname{english}{\lstlistlistingname}{List of Listings}

% -- Sonstige Pakete
\usepackage{lipsum}				% Paket für Blindtext
\usepackage{multicol}			% Mehrere Spalten auf einer Seite 
\usepackage{pdflscape}			% Rotieren einer einzelnen PDF-Seite in der landscape Umgebung
\usepackage{isodate}			% Datum wird in der Region entsprechend geändert

\pdfcompresslevel=0				% Geringer Komprimierung dadurch verbesserte Compiletime
\pdfobjcompresslevel=0			% Geringer Komprimierung dadurch verbesserte Compiletime 
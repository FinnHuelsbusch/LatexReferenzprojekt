% % Definition des neuen Befehls für das Einfügen der Abkürzung der Einheit
% \newcommand{\acrounit}[1]{
%   \acroextra{\makebox[18mm][l]{\si[per-mode=fraction,fraction-function=\sfrac]{#1}}}
% }

\DeclareAcroProperty{unit}

\NewAcroTemplate[list]{physics}{%
  \acronymsmapT{%
    \AcroAddRow{%
      \acrowrite{short}%
      &
      \acrowrite{unit}%
      &
      \acrowrite{list}%
      &
      \acropages
        {\acrotranslate{page}\nobreakspace}%
        {\acrotranslate{pages}\nobreakspace}%
      \tabularnewline
    }%
  }%
  \acroheading
  \acropreamble
  \par\noindent
  \setlength\LTleft{0pt}%
  \setlength\LTright{0pt}%
  \begin{longtable}{@{}lll@{\extracolsep{\fill}}l@{}}
    \AcronymTable
  \end{longtable}
}

\DeclareAcronym{f}{
  short = \ensuremath{f} ,
  long  = Frequenz ,
  unit  = \si{\hertz} ,
  tag   = physics
}
\DeclareAcronym{A}{
  short = \ensuremath{A} ,
  long  = Fläche ,
  unit  = \si{\metre^2} ,
  tag   = physics
}
\DeclareAcronym{C}{
  short = \ensuremath{C} ,
  long  = Kapazität ,
  unit  = \si{\farad} ,
  tag   = physics
}
\DeclareAcronym{F}{
  short = \ensuremath{F} ,
  long  = Kraft ,
  unit  = \si{\newton} ,
  tag   = physics
}

% \begin{acronym}[dmin] % längstes Kürzel wird verw. für den Abstand zw. Kürzel u. Text

% 	% Alphabetisch selbst sortieren - nicht verwendete Formeln rausnehmen!
% 	% Allgemein: \acro{KÜRZEL}[ABKÜRZUNG]{\acrounit{SI-EINHEIT}BESCHREIBUNG}

% 	\acro{A}[\ensuremath{A}]{\acrounit{mm^2}Fläche}	
% 	\acro{D}[\ensuremath{D}]{\acrounit{mm}Werkstückdurchmesser}	
% 	\acro{dmin}[\ensuremath{d\textsubscript{min}}]{\acrounit{mm}kleinster Schaftdurchmesser}	
% 	\acro{L1}[\ensuremath{L\textsubscript{1}}]{\acrounit{mm}Länge des Werkstückes Nr. 1}	
% 	\acro{Fwinkel}[]{\acrounit{Grad}Freiwinkel}	
% 	\acro{Kwinkel}[]{\acrounit{Grad}Keilwinkel}
% \end{acronym}
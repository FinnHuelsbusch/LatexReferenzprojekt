% Präambel mit Angaben zum Dokument
\documentclass[
fontsize=12pt,          	 % Leitlinien sprechen von Schriftgröße 12.
paper=A4,
twoside=false,
listof=totoc,            		% Tabellen- und Abbildungsverzeichnis ins Inhaltsverzeichnis
index=totoc,				  %Stichwortverzeichnis wird in das Inhaltsverzeichniss aufgenommen
bibliography=totoc,		  % Literaturverzeichnis ins Inhaltsverzeichnis aufnehmen
titlepage,               % Titlepage-Umgebung anstatt \maketitle
headsepline,             % horizontale Linie unter Kolumnentitel
abstracton,              % Überschrift einschalten, Abstract muss in {abstract}-Umgebung stehen
ngerman
]{scrreprt}                  % Verwendung von KOMA-Report

\usepackage[onehalfspacing]{setspace}        % Zeilenabstand \singlespacing, \onehalfspaceing, \doublespacing
\usepackage{scrlayer-scrpage} 	   % Gestaltung von Fuß- und Kopfzeilen

\usepackage{titletoc}          			  % Anpassungen am Inhaltsverzeichnis
\contentsmargin{0.75cm}       		  % Abstand im Inhaltsverzeichnis zw. Punkt und Seitenzahl

% Paket für Quellen
\usepackage[
backend = biber,                % Verweis auf biber
language = auto,
style = numeric,                % Nummerierung der Quellen mit Zahlen
sorting = none,                 % none = Sortierung nach der Erscheinung im Dokument
sortcites = true,               % Sortiert die Quellen innerhalb eines cite-Befehls
block = space,                  % Extra Leerzeichen zwischen Blocks
hyperref = true,                % Links sind klickbar auch in der Quelle
%backref = true,                % Referenz, auf den Text an die zitierte Stelle
bibencoding = auto,
giveninits = true,              % Vornamen werden abgekürzt
doi=false,                      % DOI nicht anzeigen
isbn=false,                     % ISBN nicht anzeigen
alldates=short                  % Datum immer als DD.MM.YYYY anzeigen
]{biblatex}
\addbibresource{content/sources.bib}
\setcounter{biburlnumpenalty}{3000}     % Umbruchgrenze für Zahlen
\setcounter{biburlucpenalty}{6000}      % Umbruchgrenze für Großbuchstaben
\setcounter{biburllcpenalty}{9000}      % Umbruchgrenze für Kleinbuchstaben
\DeclareNameAlias{default}{family-given}  % Nachname vor dem Vornamen
\AtBeginBibliography{\renewcommand{\multinamedelim}{\addslash\space
	}\renewcommand{\finalnamedelim}{\multinamedelim}}  % Schrägstrich zwischen den Autorennamen
\DefineBibliographyStrings{german}{
	urlseen = {Einsichtnahme:},                      % Ändern des Titels von "besucht am"
}
\usepackage[babel,german=quotes]{csquotes}         % Deutsche Anführungszeichen + Zitate


% Unterstützung für Deutsch einbinden
\usepackage[ngerman]{babel}

% Pakete für die Darstellung von mathematischen Symbolen und Eingabemöglichkeit von Umlauten
\usepackage[T1]{fontenc}
\usepackage[utf8]{inputenc}
\usepackage[activate]{microtype}       % Trennung von Wörtern wird besser umgesetzt
\usepackage{lmodern}         % Nicht-gerasterte Schriftarten (bei MikTeX erforderlich)

% Pakete für Mathematik
\usepackage[sumlimits, intlimits]{amsmath}  % Bei Integralen und Summen werden die Zahlen oben drüber geschrieben 
\usepackage{amsmath}    % Erweiterung vom Mathe-Satz
\usepackage{amssymb}
\usepackage{amsfonts}
\usepackage{MnSymbol}   % Für Symbole, die in amssymb nicht enthalten sind.
\usepackage{array}
\usepackage{float}
\usepackage{mathtools}
\usepackage{mathrsfs}

% Pakete für Abkürzungen
\usepackage[printonlyused]{acronym}	%Abkürzungen werden nur gedruckt wenn sie gebraucht wurden

% Pakete für Tabellen
\usepackage{booktabs}  % Für schönere Tabellen. Enthält neue Befehle wie \midrule
\usepackage{multirow}  % Mehrzeilige Tabellen
\usepackage{siunitx}   % Für SI Einheiten und das Ausrichten Nachkommastellen
\sisetup{locale=DE, range-phrase={~bis~}, output-decimal-marker={,}} % Damit ein Komma und kein Punkt verwendet wird.
\usepackage{xfrac} % Für siunitx Option "fraction-function=\sfrac"
\setlength{\tabcolsep}{10pt}					% Abstand des Inhalts zur Linie
\renewcommand{\arraystretch}{1.25}		%Abstand der Zeilen wird erhöht 1.0 ist default

% Pakete für Bilder
\usepackage{wrapfig}
\usepackage{graphicx}
\usepackage{caption}

% Pakete für Definitionsboxen
\usepackage{amsthm}                     % Liefert die Grundlagen für Theoreme
\usepackage[framemethod=tikz]{mdframed} % Boxen für die Umrandung
\input{content/00_LateX/highlightBoxes}  % Weitere Details sind ausgelagert

% Pakete Für Todo Notes
\usepackage{todonotes}
\setlength {\marginparwidth }{2cm}      % Abstand für Todo Notizen

% Pakete für Aufzählungen
\usepackage{enumitem}

% Pakete für Referenzen
\usepackage{varioref}
\usepackage[                        % Klickbare Links (enth. auch "nameref", "url" Package)
hidelinks,                    			 % Blende die "URL Boxen" aus.
breaklinks=true              	    % Breche zu lange URLs am Zeilenende um
]{hyperref}

\hypersetup{%
	linktocpage=true, 		% Nicht der Text sondern die Seitenzahlen in Verzeichnissen klickbar
	bookmarksnumbered=true 	% Überschriftsnummerierung im PDF Inhalt anzeigen.
}
\usepackage[noabbrev]{cleveref}   %Kein erkennen von Abkürzungen 

% Quellcodevorlage
\usepackage{scrhack}                    % Hack zur Verw. von listings in KOMA-Script
\usepackage{listings}                   % Darstellung von Quellcode
\usepackage{xcolor}                     % Einfache Verwendung von Farben
\input{content/00_LateX/sourceCodeStyle}  % Weitere Details sind ausgelagert
\usepackage{algorithm}                  % Für Algorithmen-Umgebung (ähnlich wie lstlistings Umgebung)
\usepackage{algpseudocode}              % Für Pseudocode. Füge "[noend]" hinzu, wenn du kein "endif",
% etc. haben willst.
\makeatletter                           % Sorgt dafür, dass man @ in Namen verwenden kann.
% Ansonsten gibt es in der nächsten Zeile einen Compilefehler.
\renewcommand{\ALG@name}{Algorithmus}   % Umbenennen von "Algorithm" im Header der Listings.
\makeatother                            % Zeichen wieder zurücksetzen
\renewcommand{\lstlistingname}{Listing} % Erlaubt das Umbenennen von "Listing" in anderen Titel.

% Anhänge werden sortiert
\usepackage[titletoc]{appendix} %Fügt "Anhang" in TOC ein. 

% Paket für ein Stichwortverzeichnis
\usepackage{makeidx}
\makeindex

% Nur ein latex-Durchlauf für die Aktualisierung von Verzeichnissen nötig
\usepackage{bookmark} 

% Zeilenumbruch, Seitenränder und mehr
\usepackage[
%showframe,                % Ränder anzeigen lassen
left=2.7cm, right=2.5cm,
top=2.5cm,  bottom=2.5cm,
includeheadfoot
]{geometry}                 

% Schriftart wird geändert
\usepackage{lmodern}
\renewcommand*\familydefault{\sfdefault} 				%Serifenlose Schrift 

%Unterüberschriften 
\usepackage[list=true]{subcaption} %Unterbilder werden im Abbildungsverzeichniss dargestellt 


% Bildunterschriften werden etwas kleiner dargestellt
\addtokomafont{caption}{\small}

\addto\extrasngerman{%
	\def\equationautorefname~#1\null{Gleichung~(#1)\null}
	\def\lstnumberautorefname{Zeile}
	\def\lstlistingautorefname{Listing}
	\def\algorithmautorefname{Algorithmus}
	% Damit einheitlich "Abschnitt 1.2[.3]" verwendet wird und nicht "Unterabschnitt 1.2.3"
	% \def\subsectionautorefname{Abschnitt}
}

% ---- Abstand verkleinern von der Überschrift 
\renewcommand*{\chapterheadstartvskip}{\vspace*{.5\baselineskip}}

% Einzelne Wörter
\widowpenalty10000
\clubpenalty10000

% Paket für Blindtext
\usepackage{lipsum}

% Layout für die Absätze
\setlength{\parindent} {0.0em}
\setlength{\parskip} {1.5ex plus0.5ex minus0.5ex}   % Setzt den Abstand zwischen 2 Absätzen 


% Änderungen an Schreibweisen und Silbentrennungen
% ---- Hilfreiches
\newcommand{\zB}{z.\,B. }   % "z.B." mit kleinem Leeraum dazwischen (ohne wäre nicht korrekt)
\newcommand{\dash}{d.\,h. }

% ---- Silbentrennung (falls LaTeX defaults falsch / nicht gewünscht sind)
\hyphenation{HANA}         % anstatt HA-NA
\hyphenation{Graph-Script} % anstatt GraphS-cript

% Allgemeine Informationen zur Arbeit
\newcommand{\titel}{Beispieltitel, welcher am Besten über zwei Zeilen verläuft}
\newcommand{\titelheader}{Titel welcher im Header auftaucht}
\newcommand{\arbeit}{Projektarbeit 1 (T1\_1000)}

% Persönliche Angaben und DHBW
\newcommand{\vorname}{Max}
\newcommand{\nachname}{Mustermann}
\newcommand{\matrikelnr}{1234567}
\newcommand{\kurs}{AB12CDE}
\newcommand{\studiengang}{Musterstudiengang}
\newcommand{\studienjahr}{2019}
\newcommand{\abschluss}{Bachelor of Muster}
\newcommand{\betreuerDhbw}{DH-Vorname DH-Nachname}

% Angaben zum Zeitraum
\newcommand{\bearbeitungsmonat}{Februar 2020}
\newcommand{\abgabeOrt}{Musterort}
\newcommand{\abgabeDatum}{08. September 2020}
\newcommand{\bearbeitungszeitraum}{24.03.2020 - 08.09.2020}

% Angaben zu der Ausbildungsfirma
\newcommand{\firmaName}{Musterfirma}
\newcommand{\firmaStrasse}{Musterstraße 4711}
\newcommand{\firmaPlz}{00000 Musterort, Deutschland}
\newcommand{\betreuerFirma}{B-Vorname B-Nachname}

%Optionen 
\setboolean{durchgehendeFussnoten}{true}
\setboolean{sperrvermerkLangeVersion}{true}
\ifthenelse{\boolean{durchgehendeFussnoten}}
{
	\usepackage{chngcntr}
	\counterwithout{footnote}{chapter}}
{}

% Kopf- und Fußzeilen werden eingelesen
\input{content/00_LaTeX/headerAndFooter}

\begin{document}
	\setlength{\parindent}{0pt}              	% Keine Paragraphen Einrückung.
	\setcounter{secnumdepth}{2}             % Nummerierungstiefe fürs Inhaltsverzeichnis
	\setcounter{tocdepth}{1}             	    % Tiefe des Inhaltsverzeichnisses. Ggf. so anpassen, dass das Verzeichnis auf eine Seite passt.
	\sffamily                                				% Serifenlose Schrift verwenden.
	
	% Einbinden des Deckblatts
	\singlespacing
    \thispagestyle{empty}
\begin{titlepage}
	\enlargethispage{4cm}
	\begin{figure}
		% \vspace*{-5mm} 
		\begin{minipage}{0.49\textwidth}
			\flushleft
			%\includegraphics[height=2.5cm]{images/logos/} 
		\end{minipage}
		\hfill
		\begin{minipage}{0.49\textwidth}
			\flushright
			\includegraphics[height=2.5cm]{images/logos/dhbw.png} 
		\end{minipage}
	\end{figure} 
	\vspace*{0.1cm}
	\begin{center}
		\huge{\textbf{\titel}}\\[1.5cm]
		\Large{\textbf{\arbeit}}\\[0.5cm]
		\normalsize{im Rahmen der Prüfung zum\\[1ex] \textbf{Bachelor of Muster}}\\[0.5cm]
		\Large{des Studienganges \studiengang}\\[1ex]
		\normalsize{an der Dualen Hochschule Baden-Württemberg Stuttgart}\\[1cm]
		\normalsize{von}\\[1ex] \Large{\textbf{\autor}} \\[1cm]
		\normalsize{\bearbeitungsmonat}\\[2.25cm]
		\begin{spacing}{0.8}
			\begin{tabular}{ll}
				\textbf{Abgabedatum}				\hspace{4.5cm}					& \abgabeDatum\\[0.2cm]
				\textbf{Bearbeitungszeitraum}       				&  \bearbeitungszeitraum\\[0.2cm]
				\textbf{Matrikelnummer, Kurs} 					   	&  \matrikelnr, \kurs\\[0.2cm]
				\textbf{Ausbildungsfirma}              					 &  \firmaName\\
																						& \firmaStrasse \\
																				    	& \firmaPlz\\[0.2cm]
				\textbf{Betreuer der Ausbildungsfirma}          &  \betreuerFirma\\[0.2cm]
				\textbf{Gutachter der Dualen Hochschule}    &  \betreuerDhbw\\[0.2cm]
			\end{tabular} 
		\end{spacing}
	\end{center}
\end{titlepage}

    \newcounter{savepage}
    \pagenumbering{Roman} % Seitenzahlen werden durch römische Buchstaben gekennzeichnet
    \onehalfspacing
    
      % Einbinden der Erklärung
    % Current file is content/02_declaration/declaration.tex

\chapter*{Erklärung}
	Ich versichere hiermit, dass ich meine \arbeit{} mit dem Thema:
	\begin{quote}
		\textit{\titel}
	\end{quote}
	selbstständig verfasst und keine anderen als die angegebenen Quellen und Hilfsmittel benutzt habe. Ich versichere zudem, dass die eingereichte elektronische Fassung mit der gedruckten Fassung übereinstimmt. 
	\vspace{1cm}
	
	\abgabeOrt, den \abgabeDatum \\[0.5cm]
	{\makebox[6cm]{\hrulefill}}\\ 
	\nachname , \vorname
    
    % Einbinden des Sperrvermerks
    \chapter*{Sperrvermerk}
\ifthenelse{\boolean{sperrvermerkLangeVersion}}
{
	Die vorliegende {\arbeit} mit dem Titel: 
	\begin{quote}
		\textit{\titel}
	\end{quote}
	enthält unternehmensinterne bzw. vertrauliche Informationen der {\firmaName}, ist deshalb mit einem Sperrvermerk versehen und wird ausschließlich zu Prüfungszwecken am Studiengang {\studiengang} der Dualen Hochschule Baden-Württemberg {\abgabeOrt} vorgelegt. Sie ist ausschließlich zur Einsicht durch den zugeteilten Gutachter, die Leitung des Studiengangs und ggf. den Prüfungsausschuss des Studiengangs bestimmt.  Es ist untersagt,
	\begin{itemize}
		\item den Inhalt dieser Arbeit (einschließlich Daten, Abbildungen, Tabellen, Zeichnungen usw.) als Ganzes oder auszugsweise weiterzugeben,
		\item Kopien oder Abschriften dieser Arbeit (einschließlich Daten, Abbildungen, Tabellen, Zeichnungen usw.) als Ganzes oder in Auszügen anzufertigen,
		\item diese Arbeit zu veröffentlichen bzw. digital, elektronisch oder virtuell zur Verfügung zu stellen. 
	\end{itemize}
	Jede anderweitige Einsichtnahme und Veröffentlichung – auch von Teilen der Arbeit – bedarf der vorherigen Zustimmung durch den Verfasser und {\firmaName}.
}
{
	Der Inhalt dieser Arbeit darf weder als Ganzes noch in Auszügen Personen
	außerhalb des Prüfungsprozesses und des Evaluationsverfahrens zugänglich gemacht werden,
	sofern keine anderslautende Genehmigung der Ausbildungsstätte vorliegt.
}
    
     % Einbinden des Abstracts
    \include{content/04_abstract/abstract-de}
    \include{content/04_abstract/abstract-en}
    
    % Inhaltsverzeichnis
    \singlespacing
    \tableofcontents

	% Verzeichnisse
	\renewcommand*{\chapterpagestyle}{plain}
	\pagestyle{plain}
	%Currently no working solution due to package update
\DeclareAcronym{f}{
  short = \ensuremath{f} ,
  long  = Frequenz
}
\DeclareAcronym{A}{
  short = \ensuremath{A} ,
  long  = Fläche
}
\DeclareAcronym{C}{
  short = \ensuremath{C} ,
  long  = Kapazität
}
\DeclareAcronym{F}{
  short = \ensuremath{F} ,
  long  = Kraft
}

	% \begin{acronym}[WYSISWG] % längstes Kürzel wird verw. für den Abstand zw. Kürzel u. Text
	
	% Alphabetisch selbst sortieren - nicht verwendete Kürzel rausnehmen!
	\DeclareAcronym{AIR}{
  		short = AIR,
  		long  = Adobe Integrated Runtime
	}
	% \acro{AJAX}{Asynchronous Javascript and XML}
	\DeclareAcronym{AJAX}{
  		short = AJAX,
  		long  = Asynchronous Javascript and XML
	}
	% \acro{ANSI}{American National Standards Institute}
	% \acro{API}{Application Programming Interface}
	\DeclareAcronym{API}{
		short = API,
		long  = Application Programming Interface
	}
	% \acro{AR}{Augmented Reality}
	% \acro{BAPI}{Business Application Programming Interface}
	% \acro{BIOS}{Basic Input Output System}
	% \acro{CDMA}{Code Division Multiple Access}
	% \acro{HTTPS}{Hypertext Transfer Protocol Secure}
	\DeclareAcronym{HTTPS}{
		short = HTTPS,
		long  = Hypertext Transfer Protocol Secure
	}
	% \acro{ISBN}{Internationale Standardbuchnummer}
	\DeclareAcronym{ISBN}{
		short = ISBN,
		long  = Internationale Standardbuchnummer
	}
	% \acrodefplural{ISBN}[ISBNs]{Internationale Standardbuchnummern}
	% \acro{OLAP}{Online Analytical Processing}
	% \acro{ORDBMS}{Object-Relational DataBase Management System}
	% \acro{SDK}{Software Development Kit}
	% \acro{SEO}{Search Engine Optimization}
	% \acro{SSH}{Secure Shell}
	% \acro{UEFI}{Unified Extensible Firmware Interface}
	% \acro{USB}{Universal Serial Bus}
	% \acro{VLAN}{Virtual Local Area Network}
	% \acro{WYSISWG}{What You See Is What You Get}
	% \acro{XSL}{Extensible Stylesheet Language}
	
% \end{acronym}
	\listoffigures                         													 % Erzeugen des Abbildungsverzeichnisses 
	\listoftables                          													 % Erzeugen des Tabellenverzeichnisses
    \renewcommand{\lstlistlistingname}{Quellcodeverzeichnis}
    \lstlistoflistings                     													 % Erzeugen des Listenverzeichnisses
    \setcounter{savepage}{\value{page}}

    
    % Einbinden von den verschiedenen Kapiteln
    \cleardoublepage
    \pagenumbering{arabic}                										  % Arabische Seitenzahlen für den Hauptteil
    \setlength{\parskip}{0.5\baselineskip}  								% Abstand zwischen Absätzen
    \rmfamily
    \renewcommand*{\chapterpagestyle}{scrheadings}
    \pagestyle{scrheadings}
   	\onehalfspacing
   	\sffamily                               												 % Serifenlose Schrift verwenden.
   	\include{content/06_chapter/01_einleitung}
   	\include{content/06_chapter/02_abbildungen}
   	\include{content/06_chapter/03_formatText}
   	\include{content/06_chapter/04_mathematische-formeln}
   	\include{content/06_chapter/05_quellcode}
   	\include{content/06_chapter/06_literaturHinweis}
	
	% Einbinden des Literaturverzeichnisses 
	\cleardoublepage
	\renewcommand*{\chapterpagestyle}{plain}
	\pagestyle{plain}
	\pagenumbering{Roman}                   % Römische Seitenzahlen
	\setcounter{page}{\numexpr\value{savepage}+1}
	\printbibliography[title=Literaturverzeichnis]
	
    % Einbinden von Anhängen
    \begin{appendices}
    	\include{content/07_appendix/index}
    	\include{content/06_chapter/07_Appendix}
    \end{appendices}
\end{document}